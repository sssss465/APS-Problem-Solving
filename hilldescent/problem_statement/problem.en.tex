\problemname{Hill Descent}

John is an Olympic Skier. He is currently preparing for a competition at a skiing resort and wants to maximize his efficiency of training. He gets his hands on an altitude mapping of the local mountains represented as a grid of altitudes. Given this mapping, he wants to know what is the longest descent that he could achieve. Note that John can travel North, East, South, or West and can only ski downhill, meaning he can travel to one of the four adjacent positions if that position is of strictly lower altitude. Since these paths are not official slopes, John would like a set of directions that would allow him to find his way along the longest path of descent. John also loves admiring the views from high altitudes, so if more than one longest descent exists, he would like to choose one that starts at the highest point possible.


\section*{Input}
On the first line are specified the values $n, m (1 \leq n, m \leq 1000)$, the dimensions of the grid.
Each of the next $n$ lines has $m$ integers describing the altitudes $a_{ij} (1 \leq a_{ij} \leq 1000000)$.

\section*{Output}
On the first line output the starting position of the descent. Output the row followed by the column at which the starting position is located. If there are multiple starting coordinates that allow a descent of maximum length, choose the one that starts at the highest altitude. In case of ties of highest altitude, choose the one that is more to the North. In case of ties again, choose the one that is most to the West. In other words, choose
the starting position that gives the descent of maximal length starting at the highest altitude -- in case of a tie of the highest altitude, choose the cell with a smaller row index and break the tie by a smaller column index.
On the second line output a sequence of letters that describe the direction John should take to go from the starting position to the ending position of the path. For each direction output $N, E, W, S$ for North, East, West and South respectively (North is up and East is right in the grid). Note that there might be multiple paths of maximal lengths starting at the same point. If that is the case, output the directions that would yield the lowest lexicographical ordering.
If no descent exists in the grid output "no descent".
