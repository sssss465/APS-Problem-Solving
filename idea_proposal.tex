\documentclass[a4paper,11pt]{article}
\usepackage[american]{babel}
\usepackage[utf8]{inputenc}
\usepackage{geometry}

\usepackage{booktabs}  
\usepackage{graphicx} 
\usepackage{listings}
\lstset{%
backgroundcolor=\color{cyan!10},
basicstyle=\ttfamily,
numbers=left,numberstyle=\scriptsize
}
\usepackage{hyperref}

\title{Hill Descent}
\author{Andrew Huang (amh877) \\
Kevin Li (kl2482) \\
Titas Geryba(tg1404)}

\begin{document}

\maketitle

\section{Problem Description}

John is an Olympic Skier. In order to maximize his fun on a given slope, he wants to find the longest path down the slope. has decided to maximize his distance down the hill. Help John find the longest decreasing path from the mountain down the hill. 

Input: You are given $n, m (1 \leq n, m \leq 5000)$ and then you are given $n $ rows and $m$ columns of numbers. 

Output: The length of the longest path.

%Given an array with $n$ ($1 \leq n \leq 10^5$) integers, find the maximum value among the integers.
%Integers in the array are no larger than $10^9$ in their absolute values.


\section{Solutions}

%Describe your intended solution(s) to this problem and its(their) time complexity.
There are several ways to solve this problem, ranging to naive to slightly more difficult.
For example,

\begin{itemize}
    \item Naively perform dfs on every spot and record the furthest distance reached. \\
Time complexity is $O((nm)^2)$.
    \item Use dynamic programming to consider a cell up to $i , j$ and whether or not taking a step in any valid direction would add to the  distance. So the state would be current distance, $i, j$. $O(mn)$
    \item Find a topological sort and find the length of the longest sub-length. $O(mn)$
\end{itemize}


\section{Anti-Solutions}

%Describe some anti-solution(s) that people may try and briefly explain why they will fail.
%For example,

Some common pitfalls are: 

\begin{itemize}
    \item One may think that the length of the longest path may simply be the furthest DFS traversal from the highest point, that may not be true because it may be surrounded by barriers (values larger than it). 
\end{itemize}

\section{Test Case Design}

%Provide a plan for test case generation.
Ideas for test case generation.\\
\begin{itemize}
    \item Three cases are generated randomly
    \item One case contains only negative numbers
    \item One case contains only zeroes
    \item One case with one peak and valleys
    \item Randomly generated 2D Perlin noise
\end{itemize}

A total of $7$ secret cases will be generated.

\end{document}